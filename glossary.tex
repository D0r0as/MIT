\documentclass[14 pt]{extarticle}
\usepackage[T2A]{fontenc}
\usepackage[english, russian]{babel}
\usepackage[utf8]{inputenc}
\usepackage{indentfirst}
\usepackage{titling}
\usepackage[left=2cm,right=2cm,
    top=2cm,bottom=2cm,bindingoffset=0cm]{geometry}
    
\title{Глоссарий}
\author{Дорохина Анастасия}
\date{151 гр.}

\begin{document}

\maketitle

\section*{A}
    1. \textbf{AlphaZero} (Александр Кузьмин, neoflex) — нейронная сеть, разработанная компанией DeepMind, которая использует обобщенный подход AlphaGo Zero.
\section*{B}
    2. \textbf{Backend} (Никита Барабанов, neoflex) - программно-аппаратная часть сервиса, отвечающая за функционирование его внутренней части.\\

    3. \textbf{Backlog} (Никита Барабанов, neoflex) - это список задач или требований к продукту, которые нужно реализовать.\\
    
    4. \textbf{Big Data} (Александр Кузьмин, neoflex) - это структурированные или неструктурированные массивы данных большого объема.
    
\section*{С}
    5. \textbf{ChatGPT} (Никита Рыданов, СГУ) - \\
    
\section*{D}
    6. \textbf{Data sience} (Александр Кузьмин, neoflex) - это междисциплинарная область на стыке статистики, математики, системного анализа и машинного обучения, которая охватывает все этапы работы с данными. \\
    
    7. \textbf{Deploy} (Никита Барабанов, neoflex) - это размещение готовой версии программного обеспечения на платформе, доступной для пользователей. \\

    8. \textbf{Deep learning} (Александр Кузьмин, neoflex) - совокупность методов машинного обучения, основанных на обучении представлениям, а не специализированных алгоритмах под конкретные задачи. \\

    9. \textbf{Deep mining} (Александр Кузьмин, neoflex) - собирательное название, используемое для обозначения совокупности методов обнаружения в данных ранее неизвестных, нетривиальных, практически полезных и доступных интерпретации знаний, необходимых для принятия решений в различных сферах человеческой деятельности. \\

    10. \textbf{DevOps} (Никита Барабанов, neoflex) - методология автоматизации технологических процессов сборки, настройки и развёртывания программного обеспечения. Методология предполагает активное взаимодействие специалистов по разработке со специалистами по информационно-технологическому обслуживанию и взаимную интеграцию их технологических процессов друг в друга для обеспечения высокого качества программного продукта.

\section*{E}
    11. \textbf{ERP} (Ростислав. Сибинтек) - это организационная стратегия интеграции производства и операций, управления трудовыми ресурсами, финансового менеджмента и управления активами, ориентированная на непрерывную балансировку и оптимизацию ресурсов предприятия посредством специализированного интегрированного пакета прикладного программного обеспечения (ПО), обеспечивающего общую модель данных и процессов для всех сфер деятельности.

\section*{F}
    12. \textbf{Framework} (Александр Кузьмин, neoflex) - программная платформа, определяющая структуру программной системы; программное обеспечение, облегчающее разработку и объединение разных компонентов большого программного проекта.
    
    13. \textbf{Frontend} (Никита Барабанов, neoflex) - \\

\section*{G}
    14. \textbf{GPT-модель} (Никита Рыданов, СГУ) - \\
    
    15. \textbf{GAN} (Александр Кузьмин, neoflex) - \\
    
\section*{M}
    16. \textbf{MVC модуль} (Иван Жадаев, Мастер софт) -  \\
    
    17. \textbf{Middle fullstack} (Павел, тинькофф) -  \\
    
    18. \textbf{MRP} (Ростислав. Сибинтек) - \\
    
    19. \textbf{Message box} (Никита, КРЭТ КБПА) - \\
    
\section*{O}
    20. \textbf{Open source} (Никита Рыданов, СГУ) - \\
    
\section*{P}
    21. \textbf{Pipeline} (Никита Барабанов, neoflex) - \\
    
    22. \textbf{Proxi} (Игорь Юрин, СГУ) - \\
    
    23. \textbf{Pattern} (Иван Жадаев, Мастер софт) -  \\
    
\section*{R}
    24. \textbf{Roadmap} (Павел, тинькофф) -  \\
    
\section*{S}
    25. \textbf{Soa} (Никита Барабанов, neoflex) - \\
    
\section*{T}
    26. \textbf{Team Lead} (Александр Кузьмин, neoflex) - \\
    
\section*{U}
    27. \textbf{UML} (Максим, КРЭТ КБПА) - \\

\section*{W}
    28. \textbf{Workflow} (Ростислав. Сибинтек) - \\

\section*{А}
    29. \textbf{Архитектура трансформера} (Никита Рыданов, СГУ) - \\
    
    30. \textbf{Атака} (Алёна Коноплева, ЦОПП) - преднамеренное действие злоумышленника, использующие уязвимости информационной системы и приводящие к нарушению доступности, целостности и конфиденциальности обрабатываемой информации.\\
    
    31. \textbf{Аппроксимация} (Александр Кузьмин, neoflex) - это метод вычислений, используемый в математике, заключающийся в том, что сложные математические объекты при расчетах заменяются более простыми. \\

    32. \textbf{Аутсорсинг} (Павел, тинькофф) - это процесс, при котором часть работы компании выполняется за плату другой компанией. \\
    
\section*{Б}
    33. \textbf{Баг} (Никита Барабанов, neoflex) - \\
    
    34. \textbf{Биткоин} (Игорь Юрин, СГУ) - \\
    
\section*{В}
    35. \textbf{Веб-сервер} (Игорь Юрин, СГУ) - \\
    
    36. \textbf{Вычислительная сложность} (Надежда Дёмина, ) - \\
    
\section*{Д}
    37. \textbf{Декомпозирование} (Александр Кузьмин, neoflex) - \\
    
\section*{И}
    38. \textbf{Интернет вещей} (Игорь Юрин, СГУ) - \\
    
\section*{К}
    39. \textbf{Кластеризация} (Александр Кузьмин, neoflex) - \\
    
    40. \textbf{Компьютерная безопасность} (Игорь Юрин, СГУ) - \\
    
    41. \textbf{Конечный автомат} (Леонид, ЦОПП) -  \\
    
\section*{Л}
    42. \textbf{Лог} (Никита Барабанов, neoflex) - \\
    
\section*{М}
    43. \textbf{Майнинг} (Игорь Юрин, СГУ) - \\
    
    44. \textbf{Метапрограммирование} (Никита, КРЭТ КБПА) - \\
    
    45. \textbf{Микросервисы} (Никита Барабанов, neoflex) - \\
    
    46. \textbf{Монолит} (Александр Кузьмин, neoflex) - \\
    
\section*{О}
    47. \textbf{Отладочное сообщение} (Никита, КРЭТ КБПА) - \\
    
\section*{П}
    48. \textbf{Персептрон} (Александр Кузьмин, neoflex) - \\
    
\section*{Р}
    49. \textbf{Рефракторинг кода} (Александр Кузьмин, neoflex) - \\
    
\section*{С}
    50. \textbf{Свёрточная нейронная сеть} (Александр Кузьмин, neoflex) - \\
    
    51. \textbf{Сокет} (Леонид, ЦОПП) -  \\
    
    52. \textbf{Субд} (Ростислав. Сибинтек) - \\
    
\section*{Т}
    53. \textbf{Толстый клиент} (Иван Жадаев, Мастер софт) -  \\
    
    54. \textbf{Тонкий клиент} (Иван Жадаев, Мастер софт) -  \\
    
    55. \textbf{Трассируемость} (Максим, КРЭТ КБПА) - \\
    
    56. \textbf{Троян} (Игорь Юрин, СГУ) - \\
    
\section*{Ф}
    57. \textbf{Формализация} (Александр Кузьмин, neoflex) - \\
    
    58. \textbf{Форматирование диска} (Игорь Юрин, СГУ) - \\
    
\section*{Ч}
    59. \textbf{Чёрный ящик} (Максим, КРЭТ КБПА) - \\
    
\section*{Я}
    60. \textbf{Язык запросов} (Иван Жадаев, Мастер софт) -  \\
    
\end{document}

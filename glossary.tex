\documentclass[14 pt]{extarticle}
\usepackage[T2A]{fontenc}
\usepackage[english, russian]{babel}
\usepackage[utf8]{inputenc}
\usepackage{indentfirst}
\usepackage{titling}
\usepackage[left=2cm,right=2cm,
    top=2cm,bottom=2cm,bindingoffset=0cm]{geometry}
    
\title{Глоссарий}
\author{Дорохина Анастасия}
\date{151 гр.}

\begin{document}

\maketitle

\section*{A}
    1. \textbf{AlphaZero} (Алексей Кузьмин, neoflex) "--- нейронная сеть, разработанная компанией DeepMind, которая использует обобщенный подход AlphaGo Zero.
\section*{B}
    2. \textbf{Backend} (Никита Барабанов, neoflex) "--- программно-аппаратная часть сервиса, отвечающая за функционирование его внутренней части.\\

    3. \textbf{Backlog} (Никита Барабанов, neoflex) "--- это список задач или требований к продукту, которые нужно реализовать.\\
    
    4. \textbf{Big Data} (Алексей Кузьмин, neoflex) "--- это структурированные или неструктурированные массивы данных большого объема.
    
\section*{С}
    5. \textbf{ChatGPT} (Никита Рыданов, СГУ) "--- чат"=бот, основанный на новой версии нейросетевой модели GPT 3.5. Его разработала компания OpenAI (её в 2015 году основал Илон Маск, но затем отошёл от дел) в сотрудничестве с Microsoft, которая уже встроила нейросеть в свой браузер Bing. \\
    
\section*{D}
    6. \textbf{Data sience} (Алексей Кузьмин, neoflex) "--- это междисциплинарная область на стыке статистики, математики, системного анализа и машинного обучения, которая охватывает все этапы работы с данными. \\
    
    7. \textbf{Deploy} (Никита Барабанов, neoflex) "--- это размещение готовой версии программного обеспечения на платформе, доступной для пользователей. \\

    8. \textbf{Deep learning} (Алексей Кузьмин, neoflex) "--- совокупность методов машинного обучения, основанных на обучении представлениям, а не специализированных алгоритмах под конкретные задачи. \\

    9. \textbf{Deep mining} (Алексей Кузьмин, neoflex) "--- собирательное название, используемое для обозначения совокупности методов обнаружения в данных ранее неизвестных, нетривиальных, практически полезных и доступных интерпретации знаний, необходимых для принятия решений в различных сферах человеческой деятельности. \\

    10. \textbf{DevOps} (Никита Барабанов, neoflex) "--- методология автоматизации технологических процессов сборки, настройки и развёртывания программного обеспечения. Методология предполагает активное взаимодействие специалистов по разработке со специалистами по информационно-технологическому обслуживанию и взаимную интеграцию их технологических процессов друг в друга для обеспечения высокого качества программного продукта.

\section*{E}
    11. \textbf{ERP} (Ростислав, Сибинтек) "--- это организационная стратегия интеграции производства и операций, управления трудовыми ресурсами, финансового менеджмента и управления активами, ориентированная на непрерывную балансировку и оптимизацию ресурсов предприятия посредством специализированного интегрированного пакета прикладного программного обеспечения (ПО), обеспечивающего общую модель данных и процессов для всех сфер деятельности.

\section*{F}
    12. \textbf{Framework} (Алексей Кузьмин, neoflex) "--- программная платформа, определяющая структуру программной системы; программное обеспечение, облегчающее разработку и объединение разных компонентов большого программного проекта.
    
    13. \textbf{Frontend} (Никита Барабанов, neoflex) "--- разработка пользовательского интерфейса и функций, которые работают на клиентской стороне веб"=сайта или приложения. Это всё, что видит пользователь, открывая веб"=страницу, и с чем он взаимодействует.

\section*{G}
    14. \textbf{GPT-модель} (Никита Рыданов, СГУ) "--- тип нейронных языковых моделей, впервые представленных компанией OpenAI, которые обучаются на больших наборах текстовых данных, чтобы генерировать текст, схожий с человеческим. \\
    
    15. \textbf{GAN} (Алексей Кузьмин, neoflex) "--- алгоритм машинного обучения без учителя, построенный на комбинации из двух нейронных сетей, одна из которых генерирует образцы, а другая старается отличить правильные образцы от неправильных.
    
\section*{M}
    16. \textbf{MVC модуль} (Иван Жадаев, Мастер софт) "--- схема разделения данных приложения и управляющей логики на три отдельных компонента: модель, представление и контроллер "--- таким образом, что модификация каждого компонента может осуществляться независимо. \\
    
    17. \textbf{Middle fullstack} (Павел Пасеков, тинькофф) "--- программист, способный самостоятельно с нуля разработать функциональный веб"=продукт. Он знает frontend и backend, разбирается в стеке технологий, серверах, ОС, базах данных и PaaS (среда разработки и развертывания приложений в облаке). \\
    
    18. \textbf{MRP} (Ростислав, Сибинтек) "--- система планирования потребностей в материалах, одна из наиболее популярных в мире логистических концепций, на основе которой разработано и функционирует большое число микрологистических систем. \\
    
    19. \textbf{Message box} (Никита, КРЭТ КБПА) "--- окно сообщения (диалоговое окно) с текстом для пользователя. Это модальное окно, блокирующее другие действия в приложении, пока пользователь не закроет его. MessageBox может содержать текст, кнопки и символы для отображения пользователю информации и инструкций. \\
    
\section*{O}
    20. \textbf{Open source} (Никита Рыданов, СГУ) "--- программное обеспечение с открытым исходным кодом. Исходный код таких программ доступен для просмотра, изучения и изменения, что позволяет убедиться в отсутствии уязвимостей и неприемлемых для пользователя функций, принять участие в доработке самой открытой программы, использовать код для создания новых программ и исправления в них ошибок - через заимствование исходного кода, если это позволяет совместимость лицензий, или через изучение использованных алгоритмов, структур данных, технологий, методик и интерфейсов. \\
    
\section*{P}
    21. \textbf{Pipeline} (Никита Барабанов, neoflex) "--- конвейер состоит из цепочки обрабатывающих элементов, расположенных так, что выход каждого элемента является входом для следующего; название по аналогии с физическим конвейером. Обычно между последовательными элементами обеспечивается некоторый объем буферизации. \\
    
    22. \textbf{Proxi} (Игорь Юрин, СГУ) "--- промежуточный сервер между пользователем интернета и серверами, откуда запрашивается информация. По сути, прокси "--- это посредник, фильтр или шлюз, который стоит между человеком и огромными (и не всегда безопасными) данными в сети. \\
    
    23. \textbf{Pattern} (Иван Жадаев, Мастер софт) "--- повторяемая архитектурная конструкция в сфере проектирования программного обеспечения, предлагающая решение проблемы проектирования в рамках некоторого часто возникающего контекста. \\
    
\section*{R}
    24. \textbf{Roadmap} (Павел Пасеков, тинькофф) "--- последовательность важных задач, которые нужно сделать за определённый период, чтобы выпустить продукт на рынок. Чаще всего этот инструмент используют в сфере IT и маркетинга. \\
    
\section*{S}
    25. \textbf{Soa} (Никита Барабанов, neoflex) "--- метод разработки программного обеспечения, который использует программные компоненты, называемые сервисами, для создания бизнес-приложений. Каждый сервис предоставляет бизнес-возможности, и сервисы также могут взаимодействовать друг с другом на разных платформах и языках. \\
    
\section*{T}
    26. \textbf{Team Lead} (Алексей Кузьмин, neoflex) "--- специалист координирующий деятельность команды разработчиков, распределяет сферы ответственности, взаимодействует с заказчиком, планирует и организует обучение специалистов. \\
    
\section*{U}
    27. \textbf{UML} (Максим, КРЭТ КБПА) "--- язык графического описания для объектного моделирования в области разработки программного обеспечения, для моделирования бизнес"=процессов, системного проектирования и отображения организационных структур. \\

\section*{W}
    28. \textbf{Workflow} (Ростислав, Сибинтек) "--- полная или частичная автоматизация бизнес"=процессов в организации, при которой документы, информация, задачи или задания передаются от одного участника бизнес-процесса к другому для выполнения действий согласно набору руководящих правил в предусмотренное тайминг"=планом время. \\

\section*{А}
    29. \textbf{Архитектура трансформера} (Никита Рыданов, СГУ) "--- устройство функционирования модели глубокого обучения, отличается использованием самоконтроля, дифференцированным взвешиванием значимости каждой части входных (которая включает рекурсивный вывод) данных. Используется в основном в областях обработки естественного языка (NLP) и компьютерного зрения (CV). \\
    
    30. \textbf{Атака} (Алёна Коноплева, студентка КНиИТ) "--- преднамеренное действие злоумышленника, использующие уязвимости информационной системы и приводящие к нарушению доступности, целостности и конфиденциальности обрабатываемой информации.\\
    
    31. \textbf{Аппроксимация} (Алексей Кузьмин, neoflex) "--- это метод вычислений, используемый в математике, заключающийся в том, что сложные математические объекты при расчетах заменяются более простыми. \\

    32. \textbf{Аутсорсинг} (Павел Пасеков, тинькофф) "--- это процесс, при котором часть работы компании выполняется за плату другой компанией. \\
    
\section*{Б}
    33. \textbf{Баг} (Никита Барабанов, neoflex) "--- ошибка в программе или в системе, из"=за которой программа выдает неожиданное поведение и, как следствие, результат. \\
    
    34. \textbf{Биткоин} (Игорь Юрин, СГУ) "--- пиринговая платёжная система, использующая одноимённую единицу для учёта операций. Для обеспечения функционирования и защиты системы используются криптографические методы, но при этом вся информация о транзакциях между адресами системы доступна в открытом виде. \\
    
\section*{В}
    35. \textbf{Веб-сервер} (Игорь Юрин, СГУ) "--- компьютерное программное обеспечение и базовое оборудование, которое принимает запросы через HTTP (сетевой протокол, созданный для распространения веб"=контента) или его безопасный вариант HTTPS. \\
    
    36. \textbf{Вычислительная сложность} (Надежда Дёмина, предприниматель) "--- понятие в информатике и теории алгоритмов, обозначающее функцию зависимости объёма работы, которая выполняется некоторым алгоритмом, от размера входных данных. \\
    
\section*{Д}
    37. \textbf{Декомпозирование} (Алексей Кузьмин, neoflex) "--- разделение большого и сложного на небольшие простые части. При постановке задач декомпозировать "--- значит разбить абстрактную большую задачу на маленькие задачи, которые можно легко оценить. \\
    
\section*{И}
    38. \textbf{Интернет вещей} (Игорь Юрин, СГУ) "--- сеть физических устройств, которые подключены к другим устройствам и службам через Интернет или другую сеть и обмениваются с ними данными. В настоящее время в мире более миллиарда подключенных устройств, и с каждым годом их становится больше. \\
    
\section*{К}
    39. \textbf{Кластеризация} (Алексей Кузьмин, neoflex) "--- задача неконтролируемого машинного обучения, которая группирует отдельные экземпляры данных в кластеры со сходными характеристиками. \\
    
    40. \textbf{Компьютерная безопасность} (Игорь Юрин, СГУ) "--- меры безопасности, применяемые для защиты вычислительных устройств (компьютеры, смартфоны и другие), а также компьютерных сетей (частных и публичных сетей, включая Интернет). \\
    
    41. \textbf{Конечный автомат} (Леонид Сорокин, студент КНиИТ) "--- математическая абстракция, модель дискретного устройства, имеющего один вход, один выход и в каждый момент времени находящегося в одном состоянии из множества возможных. Является частным случаем абстрактного дискретного автомата, число возможных внутренних состояний которого конечно.  \\
    
\section*{Л}
    42. \textbf{Лог} (Никита Барабанов, neoflex) "--- Файл с записями о событиях в хронологическом порядке, простейшее средство обеспечения журналирования. Различают регистрацию внешних событий и протоколирование работы самой программы "--- источника записей.  \\
    
\section*{М}
    43. \textbf{Майнинг} (Игорь Юрин, СГУ) "--- деятельность по созданию новых блоков в блокчейне для обеспечения функционирования криптовалютных платформ. За создание очередной структурной единицы обычно предусмотрено вознаграждение за счёт новых (эмитированных) единиц криптовалюты и/или комиссионных сборов. \\
    
    44. \textbf{Метапрограммирование} (Никита, КРЭТ КБПА) "--- процесс создания метапрограмм для формирования или модификации текста целевой программы. Метапрограммы "--- программы, в результате работы которых формируется необходимый исходный текст целевой программы. \\
    
    45. \textbf{Микросервисы} (Никита Барабанов, neoflex) "--- веб"=сервис, отвечающий за один элемент логики в определенной предметной области. Приложение создают как комбинацию микросервисов, каждый из которых предоставляет функциональные возможности в своей предметной области. \\
    
    46. \textbf{Монолит} (Алексей Кузьмин, neoflex) "--- традиционная модель программного обеспечения, которая представляет собой единый модуль, работающий автономно и независимо от других приложений. \\
    
\section*{О}
    47. \textbf{Отладочное сообщение} (Никита, КРЭТ КБПА)  "--- системные сообщения, отправляемые чат"=ботом прямо в чат вместе с обычными сообщениями, отправляемыми по заданному сценарию. Такие сообщения содержат отладочную информацию о работе чат"=бота и позволяют увидеть, какие действия он выполняет, а какие не выполняет и проанализировать причины, по которым сценарий отработал так или иначе. \\
    
\section*{П}
    48. \textbf{Персептрон} (Алексей Кузьмин, neoflex) "--- математическая или компьютерная модель восприятия информации мозгом (кибернетическая модель мозга), предложенная Фрэнком Розенблаттом в 1958 году и впервые реализованная в виде электронной машины «Марк'=1» в 1960 году. Перцептрон стал одной из первых моделей нейросетей, а «Марк"=1» "--- первым в мире нейрокомпьютером. \\
    
\section*{Р}
    49. \textbf{Рефракторинг кода} (Алексей Кузьмин, neoflex) "--- процесс изменения кода, призванный упростить его обслуживание, понимание и расширение, при этом не изменяя его поведение. \\
    
\section*{С}
    50. \textbf{Свёрточная нейронная сеть} (Алексей Кузьмин, neoflex) "--- класс нейронных сетей, который специализируется на обработке изображений и видео. Такие нейросети хорошо улавливают локальный контекст, когда информация в пространстве непрерывна, то есть её носители находятся рядом. \\
    
    51. \textbf{Сокет} (Леонид Сорокин, студент КНиИТ) "--- название программного интерфейса для обеспечения обмена данными между процессами. Процессы при таком обмене могут исполняться как на одной ЭВМ, так и на различных ЭВМ, связанных между собой только сетью. Сокет "--- абстрактный объект, представляющий конечную точку соединения. \\
    
    52. \textbf{Субд} (Ростислав, Сибинтек) "--- совокупность программных и лингвистических средств общего или специального назначения, обеспечивающих управление созданием и использованием баз данных. СУБД "--- комплекс программ, позволяющих создать базу данных и манипулировать данными. \\
    
\section*{Т}
    53. \textbf{Толстый клиент} (Иван Жадаев, Мастер софт) "--- клиент, который сам выполняет основную часть любых операций обработки данных и не обязательно полагается на сервер. \\
    
    54. \textbf{Тонкий клиент} (Иван Жадаев, Мастер софт) "--- компьютер или программа, работающие в сетевом пространстве с терминальной или клиент"=серверной архитектурой. \\
    
    55. \textbf{Трассируемость} (Максим, КРЭТ КБПА) "--- возможность отслеживать что'=либо. В некоторых случаях это интерпретируется как способность проверять историю, местоположение или применение предмета с помощью документированной зарегистрированной идентификации. \\
    
    56. \textbf{Троян} (Игорь Юрин, СГУ) "--- разновидность вредоносной программы, проникающая в компьютер под видом легитимного программного обеспечения, в отличие от вирусов и червей, которые распространяются самопроизвольно. \\
    
\section*{Ф}
    57. \textbf{Формализация} (Алексей Кузьмин, neoflex) "---  уточнение содержания познания, осуществляемое посредством того, что изучаемые объекты, явления, процессы сопоставляются с некоторыми материальными конструкциями, позволяющими выявлять и фиксировать существенные и закономерные стороны рассматриваемых объектов. \\
    
    58. \textbf{Форматирование диска} (Игорь Юрин, СГУ) "--- Программный процесс разметки области хранения данных электронных носителей информации, расположенной на магнитной поверхности, оптических носителях. \\
    
\section*{Ч}
    59. \textbf{Чёрный ящик} (Максим, КРЭТ КБПА) "--- Стратегия тестирования функционального поведения объекта с точки зрения внешнего мира, при котором не используется знание о внутреннем устройстве тестируемого объекта. \\
    
\section*{Я}
    60. \textbf{Язык запросов} (Иван Жадаев, Мастер софт) "--- Искусственный язык, на котором делаются запросы к базам данных и информационно-поисковым системам. \\
    
\end{document}

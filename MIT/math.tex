\documentclass[14pt]{extarticle}
\usepackage[T2A]{fontenc}
\usepackage[utf8]{inputenc}
\usepackage{amsthm}
\usepackage{amsmath}
\usepackage{amssymb}
\usepackage{amsfonts}
\usepackage{mathrsfs}
\usepackage{fancyvrb}
\usepackage{indentfirst}
\usepackage[
  left=2cm, right=2cm, top=2cm, bottom=2cm, headsep=0.2cm, footskip=0.6cm, bindingoffset=0cm
]{geometry}
\usepackage[english,russian]{babel}

\begin{document}

\section*{Вариант 26}
  Уравнение математической модели:
    
  \begin{equation}
      q_0(q_0^0 + q_\text{т}^0 + K p_x^0) + q(q_0^\text{т} + q_\text{т}^\text{т} + K p_x^\text{т}) + p_0(q_0^\text{тр} + q_\text{т}^\text{тр} + K p_x^\text{тр}) = 1.
  \end{equation}
    
  По аналогии с предыдущим вариантом модели условные вероятности $q_0$(($l$ + 1)/$l$), $q$(($l$ +
  1)/$l$), $p_x$(($l$ + 1)/$l$) правильного приема нулевого, токового, стирания, соответственно, для
  ($l$ + 1)-го символа запишутся в виде:
    
  \begin{equation}\label{eq}
      \left.
        \begin{array}{cc}
          q_0((l + 1)/l) = q_0 q_0^0 + q q_0 ^ \text{т} + K p_x q_0^x; \\
          q_\text{т}((l + 1)/l) = q_0 q_\text{т}^0 + q q_\text{т}^\text{т} + K p_x q_\text{т}^x; \\
          p_0((l + 1)/l) = q_0 p_\text{тр}^0 + q p_\text{тр}^\text{т} + K p_x p_x^x.
        \end{array}
      \right\}
  \end{equation}
  Решая систему уравнений (\ref{eq}), можно получить формулы для определения безусловных вероятностей $q_0$, $q$, $p_x$.

\end{document}


